% Template for ICIP-2019 paper; to be used with:
%          spconf.sty  - ICASSP/ICIP LaTeX style file, and
%          IEEEbib.bst - IEEE bibliography style file.
% --------------------------------------------------------------------------

\documentclass{article}

% Two Column Compatible, Variable Width Algorithm Environments
% Written by Daniel Herber
% Based on modifications to the stack exchange answer at
% http://tex.stackexchange.com/questions/23296/setting-caption-rule-width-in-algorithm2e-algs/39574

% inputs
% 1 : float specifiers (htbp!)
% 2 : algorithm width (length)
% 3 : indent width (length)

% column environment
% \begin{vAlgorithm}[float specifiers]{algorithm width}{indent width}
% \end{vAlgorithm}

% page environment
% \begin{vAlgorithm*}[float specifiers]{algorithm width}{indent width}
% \end{vAlgorithm*}

% required packages
\usepackage[linesnumbered]{algorithm2e}

\SetKw{Continue}{continue}
\SetKw{Break}{break}
\SetKw{From}{from}
\SetKw{Import}{import}
\SetKw{As}{as}
\SetKw{Global}{global}
\SetKw{AndL}{and}
\SetKw{OrL}{or}

\SetKwProg{Def}{function}{}{end\ function}

\usepackage{etoolbox}
\usepackage{xstring}
\usepackage{calc}

% define lengths
\newlength{\xalgowidth}
\newlength{\xalgoremainder}
\newlength{\xindentwidth}

% define vAlgorithm* environment
\newenvironment{vAlgorithm*}[3][]{% before
  \setlength{\xalgowidth}{#2} % set algorithm width from second input
  \setlength{\xindentwidth}{#3} % set indent width from third input
  \setlength{\xalgoremainder}{\textwidth-\xalgowidth} % calculate indent to center the float
  \SetCustomAlgoRuledWidth{\xalgowidth} % set the rule width
  \IncMargin{\xindentwidth}
  \begin{algorithm*}[#1]
}% end before
{% after
  \end{algorithm*}
  \DecMargin{\xindentwidth}
}% end after

% define vAlgorithm environment
\newenvironment{vAlgorithm}[3][]{% before
  \setlength{\xalgowidth}{#2} % set algorithm width from second input
  \setlength{\xindentwidth}{#3} % set indent width from third input
  \setlength{\xalgoremainder}{\columnwidth-\xalgowidth} % calculate indent to center the float
  %\SetCustomAlgoRuledWidth{\xalgowidth} % set the rule width
  \SetCustomAlgoRuledWidth{\columnwidth} % set the rule width
  \IncMargin{\xindentwidth}
  \begin{algorithm}[#1] % column
}% end before
{% after
  \end{algorithm} % column
  \DecMargin{\xindentwidth}
}% end after

% redefine algorithm commands
\makeatletter
\patchcmd{\@algocf@start}{%
\begin{lrbox}{\algocf@algobox}%
}{%
\rule{0.5\xalgoremainder}{\z@}% indent
\begin{lrbox}{\algocf@algobox}%
\begin{minipage}{\xalgowidth}%
}{}{}
\patchcmd{\@algocf@finish}{%
\end{lrbox}%
}{%
\end{minipage}%
\end{lrbox}%
}{}{}
\makeatother


\usepackage{spconf,amsmath,graphicx}
\usepackage{multicol, blindtext} % add here
\usepackage{booktabs}
\usepackage{xcolor}
%\usepackage[linesnumbered,boxed]{algorithm2e}
\newcommand\mycommfont[1]{\footnotesize\ttfamily\textcolor{blue}{#1}}
\SetCommentSty{mycommfont}

% Example definitions.
% --------------------
\def\x{{\mathbf x}}
\def\L{{\cal L}}

% Title.
% ------
\title{SYMBOLIC PROGRAM SLICING ON SMART CONTRACTS}
%
% Single address.
% ---------------

\name{Zi Xuan Chen, Fang Yu}
\address{National Chengchi University \\
Department of Computer Science, Management Information Systems \\
104703010@nccu.edu.tw, yuf@nccu.edu.tw}

%
% For example:
% ------------
%\address{School\\
%	Department\\
%	Address}
%
% Two addresses (uncomment and modify for two-address case).
% ----------------------------------------------------------
%\twoauthors
%  {Zi Xuan Chen
%    %\sthanks{The fourth author performed the work while at ...}
%  }
%	{National Chengchi University, \\ Department of Computer Science \\
%	s10410@nccu.edu.tw}
%  {Fang Yu
%    %\sthanks{Thanks to XYZ agency for funding.}
%  }
%	{National Chengchi University, \\ Department of Management Information Systems, \\
%	yuf@nccu.edu.tw}


\begin{document}
%\ninept
%

\maketitle

%

\begin{abstract}

  %we propose a method to do program slicing. with slicing, we can analyse properties more efficiently.

  We propose a method to do program slicing on stack-based programming language. With slicing, we can analyse properties more efficiently. We take the EVM bytecode\cite{wood2014ethereum} as the example language, which is used to write smart contract\cite{szabo2003advances} on Ethereum\cite{wood2014ethereum}.

\end{abstract}

\begin{keywords}
  BlockChain, Ethereum, Slicing, Verification
\end{keywords}


\section{Introduction}
\label{sec:introduction}

There are many interest properties about Ethereum. Ethereum can be viewed as a transaction-based state machine. On Ethereum, all transaction executed on Ethereum Virtual Machine (EVM), which is a simple stack-based architecture, limits stack item size to 1024. Executions will be failed if stackoverflow occured while executing the smart contract. Besides stack, \textbf{\textit{gas}} is another interesting property on Ethereum. To limit the cost of the transaction execution, EVM takes the handling fee named \textbf{\textit{gas}} from transaction sender.

To Analyze the properties more precisely, we slice the smart contract by construcing dependencies between instructions. With the dependencies, we can slice a smaller program from some interested point. Then we can the the sliced program for other analysis purpose.

\section{Related Work}
\label{sec:relatedwork}

TBC.

\section{Method}
\label{sec:method}

To compute the dependencies between instructions, we need construct the control flow graph first. Unlike register-based machine's instruction, which's operand are called as register explicitly, for the stack-based machine, the operand that instructions depended are stored on the stack implicitly. Thus, a control flow graph for stack simulation is needed.


The first step to construct control flow graph is spliting basic blocks. We split basic blocks by \textbf{\textit{JUMPDEST}} and \textbf{\textit{end instructions}}. \textbf{\textit{JUMPDEST}} is normally considered as the beginning of blocks becuase other blocks can target \textbf{\textit{JUMPDEST}} to connect the edges. The \textbf{\textit{end instructions}} include \textbf{\textit{STOP}}, \textbf{\textit{SELFDESTRUCT}}, \textbf{\textit{RETURN}}, \textbf{\textit{REVERT}}, \textbf{\textit{INVALID}}, \textbf{\textit{SUICIDE}}, \textbf{\textit{JUMP}}, \textbf{\textit{JUMPI}}.

The second step is building the edges between basic blocks. Some edges can be computed by simply succeeding the pc of the \textbf{\textit{JUMPI}} and other instructions $\notin$ \textbf{\textit{end instructions}}, followed by \textbf{\textit{JUMPDEST}}. Because the property of the stack-based machine, all the jump destinations are pushed to stack implicitly. For this problem, we use value set analysis to find all the possible destination values.

\RestyleAlgo{algoruled} % ruled algorithm
\begin{vAlgorithm}[ht!]{0.57\textwidth}{1em}
  %{algorithm width}{indent width}
  \SetNoFillComment
  \caption{buildCFGandStackDependency}
  \KwIn{$Opcode$}
  \KwOut{$CFG, StackDG$}
  %\tcc{split blocks with $JUMP$, $JUMPI$, \\ $STOP$, $SELFDESTRUCT$, $RETURN$, \\ $REVERT$, $INVALID$, $SUICIDE$ \\ $JUMPDEST$ ... etc}
  \Def{buildCFGandStackDependency(opcode)}{
    dg = DG(opcode)\;
    cfg = CFG(opcode)\;
    cfg.buildBasicBlocks()\;
    cfg.buildSimpleEdges()\;
    cfg.buildFunctions(cfg.basicBlocks.first)\;
    \For{func $\in$ cfg.functions}{
      valueSetAnalysis(cfg, dg, func)\;
    }
    \Return cfg, dg\;
  }
  \tcc{state constructor}
  \Def{State()}{
    this.visit = dictionary(default=0)\;
    this.stacksIn = dictionary(default=None)\;
    this.stacksOut = dictionary(default=None)\;
    this.lastDiscoveredTargets = dictionary(default=$\O$)\;
    this.discoveredTargets = dictionary(default=$\O$)\;
  }
  \tcc{value set analysis}
  \Def{valueSetAnalysis(cfg, dg, func)}{
    stat = State()\;
    toExplore = \{ func.entry \}\;
    \Do{toExplore}{
      outBlocks = \{ toExplore.pop() \}\;
      \Do{outBlocks}{
        block = outBlocks.pop()\;
        outBlocks = outBlocks $\cup$ \\ \quad \quad \quad \quad \quad transFuncBlock(cfg, dg, func, \\ \quad \quad \quad \quad \quad \quad \quad \quad \quad \quad \quad \ \ \ block, stat)\;
      }
      \For{src, dsts $\in$ stat.lastDiscoveredTargets}{
        cfg.addEdges(src, dsts)\;
        toExplore = toExplore $\cup$ dsts\;
      }

      stat.lastDiscoveredTargets = dictionary(default=$\O$)\;
    }
    cfg.computeReachability(func.entry, func.id)\;
  }

\end{vAlgorithm}

\RestyleAlgo{algoruled} % ruled algorithm
\begin{vAlgorithm}[ht!]{0.56\textwidth}{1em}
  \SetNoFillComment
  %{algorithm width}{indent width}
  \caption{ValueSetAnalysisUtility}

  \tcc{trasfer function blocks}
  \Def{transFuncBlock(cfg, dg, func, block, stat)}{
    \uIf{(func.id = DISPATCHER\_ID \\ \quad \quad \quad \quad \AndL block.reacheable) \\ \quad \OrL stat.visit[block] $>\ visitLimit$}{ \Return\; }

    stat.visit[block] += 1\;

    \tcc{save prev stack to check convergence}
    prevStack, \_ = stat.stacksOut[block]

    oprdStack, instStack = abstStack(), listStack()\;

    inBlocks = [b $\in$ block.inBlocks \\ \quad \quad \quad \quad \quad \quad $|$ stat.stacksOut[b] $\neq$ None]

    \For{father $\in$ inBlocks}{
      ostk, istk = stat.stacksOut[father]\;
      oprdStack = oprdStack.merge(ostk)\;
      instStack = oprdStack.merge(istk)\;
    }
    exploreBlock(dg, block, oprdStack, instStack, stat)\;
    \uIf{block.end $\in$ \{JUMP, JUMPI\}}{
      oprdStack, \_ = stat.stacksIn[end]\;
      \For{ dst $\in$ oprdStack.top().vals()}{
        \uIf{isJumpDest(dst)}{
          addBranch(src, dst, stat)\;
        }
      }
    }
    oprdStack, \_ = stat.stacksOut[end]\;
    \uIf{prevStack $\neq$ oprdStack}{
      \tcc{ not converged }
      \Return block.outBlocksByFunc(func.id)\;
    }
    \Return $\O$;
  }

  \tcc{explore basic block}
  \Def{exploreBlock(dg, block, oprdStack, instStack, stat)}{
    \For{inst $\in$ block.instructions}{
      stat.stacksIn[inst] = (oprdStack, instStack)\;
      stat.stacksOut[inst] = \\ \quad transferFuncInst(dg, inst, \\ \quad \quad \quad oprdStack, instStack, stat)\;
    }
  }
  \tcc{add branch to value set analysis}
  \Def{addBranch(src, dst, stat)}{
    \uIf{dst $\notin$ stat.discoveredTargets[src]}{
      \uIf{src $\notin$ stat.lastDiscoveredTargets}{
        stat.lastDiscoveredTargets[src] = $\O$\;
      }
      stat.lastDiscoveredTargets[src].add(dst)\;
      stat.discoveredTargets[src].add(dst)\;
    }
  }

\end{vAlgorithm}

\RestyleAlgo{algoruled} % ruled algorithm
\begin{vAlgorithm}[ht!]{0.56\textwidth}{1em}
  \SetNoFillComment
  %{algorithm width}{indent width}
  \caption{ValueSetAnalysisUtility}
  \tcc{ transfer instruction }
  \Def{transferFuncInst(dg, inst, oprdStack, instStack, stat)}{
    oprdStack = oprdStack.copy()\;
    instStack = instStack.copy()\;
    \uIf{ inst $\in$ PUSHn[n=1..32] }{
     oprdStack.push(inst.operand)\;
     instStack.push(inst)\;
    }
    \uElseIf{ inst $\in$ SWAPn[n=1..16]}{
      oprdStack.swap(n)\;
      instStack.swap(n)\;
    }
    \uElseIf{ inst $\in$ DUPn[n=1..16]}{
      oprdStack.dup(n)\;
      instStack.dup(n)\;
    }
    \uElseIf{ inst = AND }{
      v1, v2 = oprdStack.pop(), oprdStack.pop()\;
      oprdStack.push(absAnd(v1, v2))\;
      v1s, v2s = instStack.pop(), instStack.pop()\;
      \For{ v1, v2 $\in$ zip(v1s, v2s)}{
        dg.addEdges(inst, [v1, v2])\;
      }
      instStack.push(inst)\;
    }
    \uElse{
      \RepTimes{inst.popNumber}{
        oprdStack.pop()\;
      }
      \For{args $\in$ [instStack.pop() \\ \quad \quad \ $|$ n $\in$ range(inst.popNumber)]$^T$}{
        dg.addEdges(inst, args)\;
      }
      \RepTimes{inst.pushNumber}{
        oprdStack.push(None)\;
        instStack.push(inst)\;
      }
    }
    \Return oprdStack, instStack\;
  }
\end{vAlgorithm}


\RestyleAlgo{algoruled} % ruled algorithm
\begin{vAlgorithm}[ht!]{0.56\textwidth}{1em}
  \SetNoFillComment
  %{algorithm width}{indent width}
  \caption{AnalysisEnvironment}

  \tcc{Return All dependant program counters}

  \Def{depInsts(env, inst)}{
    \Return env.addrDepInsts[inst] \\ \quad \quad \quad $\cup$ env.offsetDepInsts[inst] \\ \quad \quad \quad $\cup$ env.valDepInsts[inst]
  }

  \tcc{check insts if have same addr parameters }

  \Def{addrOverlap(env, instA, instB)}{
    rangeA = product(env.conAddrs[instA], \\ \quad \quad \quad \quad \quad \quad \quad \ \ env.conOffsets[instA])\;
    rangeB = product(env.conAddrs[instB], \\ \quad \quad \quad \quad \quad \quad \quad \ \ env.conOffsets[instB])\;
    \For{(Aa, Ao), (Ba, Bo) $\in$ product(rangeA, rangeB)}{
      \uIf{\{Aa..Aa + Ao\} $\cap$ \{Ba..Ba + Bo\} $\neq$ $\O$}{
        \Return True\;
      }
    }
    \Return False\;
  }
  \tcc{Environment constructor}

  \Def{Environment(stackDg)}{

    rInsts = stackDg.rInsts \;
    wInsts = stackDg.wInsts \;
    addrs = [ i, eval(i.addrs) $|$ i $\in$ rInsts $\cup$ wInsts ] \;
    offsets = [ i, eval(i.offsets) $|$ i $\in$ rInsts $\cup$ wInsts ] \;
    vals = [ i, eval(i.vals) $|$ i $\in$ wInsts ] \;

    \tcc{eval instructions' parameters (addr)}
    this.conAddrs = \{ i: con $|$ (i, (con, \_)) $\in$ addrs \}\;
    this.addrDepInsts = \{ i: dep $|$ (i, (\_ , dep)) $\in$ addrs \}\;

    \tcc{eval instructions' parameters (offset)}

    this.conOffsets = \{ i: con $|$ (i, (con, \_)) $\in$ offsets \}\;
    this.offsetDepInsts = \{ i: dep $|$ (i, (\_ , dep)) $\in$ offsets \}\;

    \tcc{eval instructions' parameters (val)}
    this.conVals = \{ i: con $|$ (i, (con, \_)) $\in$ vals \}\;
    this.valDepInsts = \{ i: dep $|$ (i, (\_ , dep)) $\in$ vals \}\;

    \tcc{write insts that can't be re-evaled}
    this.evaled = \{i $\in$ addrDepInsts $|$ addrDepInsts[i] = $\O$\} \\
    \quad $\cap$ \{i $\in$ offsetDepInsts $|$ offsetDepInsts[i] = $\O$\} \\
    \quad $\cap$ \{i $\in$ valDepInsts $|$ valDepInsts[i] = $\O$\}
  }

\end{vAlgorithm}

\RestyleAlgo{algoruled} % ruled algorithm
\begin{vAlgorithm}[ht!]{0.57\textwidth}{1em}
  %{algorithm width}{indent width}
  \SetNoFillComment
  \caption{buildAddressDependency}
  \KwIn{$CFG$, $StackDG$}
  \KwOut{$AddressDG$}

  \Import AnalysisEnvironment \As env

  \Def{buildAddressDependency(cfg, stackDg, opcode)}{
    \tcc{declare and alias variables}
    addrDg = DG(opcode)\;
    visit, alter = $\O$, $\O$ \;
    sreads = stackDg.SLOADs \;
    swrites = stackDg.SSTOREs \;
    mreads = stackDg.MSTOREs \;
    mwrites = stackDg.\{MLOAD $\cup$ SHA3 \\ \quad \quad \quad \quad \quad $\cup$ CREATE $\cup$ CALL $\cup$ RETURN\}s

    \tcc{new a environment}
    env = Environment(stackDg)\;
    \tcc{build addr dependency of \\ storage and memroy}
    \While{True}{
      evaled = env.evaled.copy()\;
      buildDependency(addrDg, swrites, sreads, visit)\;
      buildDependency(addrDg, mwrites, mreads, visit)\;
      \uIf{exist write inst can be re-evaled}{
        \For{inst $\in$ re-evaled}{
          update $Environment$ $variables$ in env \\ \quad \quad with eval($instruction$ $parameters$)
        }
      }
      \uIf{env.evaled $\setminus$ evaled = $\O$}{
        \Break\;
      }
    }
    \Return addrDg\;
  }

  \tcc{helper function}
  \Def{buildDependency(addrDg, writes, reads, visit)}{

    concrete = \{inst $\in$ (writes $\setminus$ visit) \\ \quad \quad \quad \quad \quad \quad \ \ $|$ \ \ $depInsts(env, inst)$ = $\O$\}\;
    \While{concrete $\neq$ $\O$}{

      \For{inst $\in$ concrete}{
        block = CFG.blockOf(inst)\;
        dfsCFG(addrDg, inst, block, writes, reads, $\O$)\;
      }
      visit.update(concrete)\;
      \For{inst $\in$ (writes $\setminus$ visit)}{
        \uIf{(depInsts(env, inst) $\setminus$ env.evaled) = $\O$}{
          update $Environment$ $variables$ in env \\ \quad \quad with eval($instruction$ $parameters$)
        }
      }

      concrete = \{ins $\in$ (writes $\setminus$ visit) \\ \quad \quad \quad \quad \quad \quad \ \ $|$ \ \ $depInsts(env, ins)$ = $\O$\}\;
    }
  }
\end{vAlgorithm}

\RestyleAlgo{algoruled} % ruled algorithm
\begin{vAlgorithm}[ht!]{0.57\textwidth}{1em}
  %{algorithm width}{indent width}
  \SetNoFillComment
  \caption{dfsCFG}

  \Import AnalysisEnvironment \As env

  \tcc{do CFG dfs for building dependency}
  \Def{dfsCFG(addrDg, wInst, block, writes, reads, visit)}{
    visit.add(block)\;
    rwInsts = block.insts $\cap$ (writes $\cup$ reads)\;
    \uIf{wInst $\in$ block}{
      rwInsts = rwInsts $\setminus$ \\ \quad \quad \{ inst $\in$ block.insts $|$ inst.pc $<$ wInst.pc\}\;
    }
    \For{inst $\in$ rwInsts}{
      \tcc{if "exist the probability" to re-write the same address then return, "probability" means the "or" part}
      \uIf{inst.name = wInst.name
        \\ \quad \AndL (addrOverlap(env, wInst, inst) \\ \quad \quad \quad \OrL env.addrDepInsts[inst] $\neq$ $\O$)}{
        visit.remove(block)\;
        \Return;
      }
      \uIf{inst $\in$ reads}{
        deps = env.addrDepInsts[inst] $\cup$ \\ \quad \quad \quad \quad \ env.offsetDepInsts[inst]\;
        \uIf{deps $\neq$ $\O$ \AndL \\
          \quad  deps $\setminus$ env.evaled = $\O$}{
          update $Environment$ $variables$ in env \\ \quad \quad with eval($instruction$ $parameters$)
        }

        \uIf{addrOverlap(env, wInst, inst)}{
          env.evaled.add(inst)\;
          addrDg.addEdge(wInst, inst)\;
          env.conVals[inst].update(\\
          \quad \quad \quad \quad \quad \quad \ \ env.conVals[wInst]);
        }
      }
    }
    \For{nextBlock $\in$ block.outBlock}{
      dfsCFG(wInst, nextBlock, writes, reads, visit)\;
    }
    visit.remove(block)\;
  }

\end{vAlgorithm}

\RestyleAlgo{algoruled} % ruled algorithm
\begin{vAlgorithm}[ht!]{0.55\textwidth}{1em}
  %{algorithm width}{indent width}
  \SetNoFillComment
  \caption{eval}
  \KwIn{$instruction$, $visit$}
  \KwOut{$concrete$ $values$, $dependant$ $PCs$}

  \Def{eval(inst, visit)}{
    \uIf{inst $\in$ visit}{
      \Return $\O$, \{inst\}\;
    }
    visit.add(inst)\;
    concrete, dependant = $\O$, $\O$\;
    \uIf{inst.name.startswith('PUSH')}{
      \Return \{int(op.operand)\}, $\O$;
    }
    cons, deps = \{map(eval(\_ , visit), argList) \\ \quad \quad \quad \quad \quad \quad \quad \quad $|$ argList $\in$ inst.argLists\}$^T$\;
    \For{argList $\in$ cons}{
      val = None\;
      \uIf{None $\in$ argList}{\Continue\;}
      \uElseIf{inst.name = 'ADD'}{
        val = \Let x, y = argList \In x + y\;
      }
      \uElseIf{inst.name = 'SUB'}{
        val = \Let x, y = argList \In x - y\;
      }
      \uElseIf{inst.name = 'MUL'}{
        val = \Let x, y = argList \In x * y\;
      }
      \uElseIf{inst.name = 'DIV'}{
        val = \Let x, y = argList \In x / y\;
      }
      \uElseIf{inst.name = 'EXP'}{
        val = \Let x, y = argList \In $x^{y}$\;
      }
      \uElseIf{inst.name = 'ISZERO'}{
        \[
        val = \Let [x] = argList \ \In
          \begin{cases}
            0,& \text{if } x = 0\\
            1,              & \text{otherwise}
          \end{cases}\;
        \]
      }
      \uElseIf{inst.name = 'NOT'}{
        val = \Let [x] = argList \In (1 $<<$ 256) - 1 - x\;
      }
      \uElseIf{inst.name = 'AND'}{
        val = \Let x, y = argList \In $x \ \& \ y$\;
      }
      \uElseIf{inst.name = 'OR'}{
        val = \Let x, y = argList \In $x \ | \ y$\;
      }
      \uElseIf{inst.name = 'EQ'}{
        val = \Let x, y = argList \In $x \ = \ y$\;
      }
      \uElseIf{inst.name $\in$ \{'MLOAD', 'SLOAD', 'SHA3'\}}{
        concrete.update(env.conVals[inst])\;
      }
      \uElse{
        \tcc{SHA3 not impl yet}
        throw Exception("not handle the inst yet")\;
      }
      \uIf{val $\neq$ None}{ concrete.add(val)\; }
      dependant.update(concat(deps))\;
    }
    visit.remove(inst)\;
    \Return concrete, dependant\;
  }

\end{vAlgorithm}

\section{experiment}
\label{sec:experiment}

\section{Conclusion}
\label{sec:conclusion}

% To start a new column (but not a new page) and help balance the last-page
% column length use \vfill\pagebreak.
% -------------------------------------------------------------------------
%\vfill
%\pagebreak

% References should be produced using the bibtex program from suitable
% BiBTeX files (here: strings, refs, manuals). The IEEEbib.bst bibliography
% style file from IEEE produces unsorted bibliography list.
% -------------------------------------------------------------------------
\bibliographystyle{IEEEbib}
\bibliography{strings,refs}

\end{document}
